% Created 2020-03-20 Fri 17:13
% Intended LaTeX compiler: pdflatex
\documentclass[10pt,AMS Euler]{article}
\usepackage[utf8]{inputenc}
\usepackage[T1]{fontenc}
\usepackage{graphicx}
\usepackage{grffile}
\usepackage{longtable}
\usepackage{wrapfig}
\usepackage{rotating}
\usepackage[normalem]{ulem}
\usepackage{amsmath}
\usepackage{textcomp}
\usepackage{amssymb}
\usepackage{capt-of}
\usepackage{hyperref}
\usepackage{minted}
\textheight=9.25in
\textwidth=7in
\topmargin=-0.75in
\oddsidemargin=-0.25in
\evensidemargin=-0.25in
\usepackage{mathtools}
\usepackage{pgfplots}
\usepackage[binary-units]{siunitx}
\usepackage{xstring}
\usepackage{mdframed}
\usepackage{amsthm}
\usepackage{nicematrix}
\NiceMatrixOptions{transparent,nullify-dots}

\allowdisplaybreaks

\usemintedstyle{xcode}

%%% NAMED SET COMMANDS
\newcommand{\RR}{\mathbb{R}}
\newcommand{\NN}{\mathbb{N}}
\newcommand{\ZZ}{\mathbb{Z}}
\newcommand{\QQ}{\mathbb{Q}}
\newcommand{\CC}{\mathbb{C}}
\newcommand{\PP}{\mathbb{P}}

%%% VECTOR COMMAND
\newcommand{\vect}[2][bold]{
  \IfEqCase{#1}{%
    {bold}{\mathbf{#2}}%
    {arrow}{\overrightarrow{#2}}%
  }%
}

\newcommand{\falling}[2]{#1^{\underline{#2}}}

%%% SHORT MATRIX COMMAND
% Prototype: MACRO mat OPT[#1={b}] OPT[#2={r}] #3
\makeatletter
\newcommand{\mat}[1][b]{%
  \@ifnextchar[{\mat@i[{#1}]}{\mat@i[{#1}][{r}]}%
}

\def\mat@i[#1][#2]#3{%
  % Put your code here.
  % You can refer to the arguments as #1 through #3.
  \IfEqCase{#1}{%
    {p}{\begin{pmatrix*}[#2]#3\end{pmatrix*}}%
    {b}{\begin{bmatrix*}[#2]#3\end{bmatrix*}}%
    {B}{\begin{Bmatrix*}[#2]#3\end{Bmatrix*}}%
    {v}{\begin{vmatrix*}[#2]#3\end{vmatrix*}}%
    {V}{\begin{Vmatrix*}[#2]#3\end{Vmatrix*}}%
    {s}{\begin{smallmatrix*}[#2]#3\end{smallmatrix*}}%
    {sp}{\begin{psmallmatrix*}[#2]#3\end{psmallmatrix*}}%
    {sb}{\begin{bsmallmatrix*}[#2]#3\end{bsmallmatrix*}}%
    {sB}{\begin{Bsmallmatrix*}[#2]#3\end{Bsmallmatrix*}}%
    {sv}{\begin{vsmallmatrix*}[#2]#3\end{vsmallmatrix*}}%
    {sV}{\begin{Vsmallmatrix*}[#2]#3\end{Vsmallmatrix*}}%
  }%[\begin{#1}#3\end{#1}]
}
\makeatother

%%% ANSWER ENVIRONMENT
\mdfdefinestyle{answerstyle}{innerrightmargin=10pt,innerleftmargin=10pt,frametitle={Answer:},linewidth=2pt}
\newenvironment{answer}{%
  \begin{mdframed}[style=answerstyle]
  \begingroup
  \setlength{\parindent}{0pt}
}{
  \endgroup
  \end{mdframed}
}
% answer environment with indentation 
\newmdenv[style=answerstyle]{answeri}

%%% EVALUATION BARS
\newcommand{\evald}{\bigg\rvert}
\newcommand{\evaldb}{\Bigg\rvert}
\newcommand{\evalds}{\Big\rvert}

%%% THEOREM ENVIRONMENTS
\newtheorem{theorem}{Theorem}[section]
\newtheorem{corollary}{Corollary}[theorem]
\newtheorem{lemma}[theorem]{Lemma}
\newtheorem{proposition}{Proposition}[section]

\theoremstyle{definition}
\newtheorem{definition}{Definition}[section]
\theoremstyle{plain}

%%% MANUALLY NUMBERED THEOREM ENVIRONMENTS
% don't care about numbering since it must be set manually
\newtheorem{innertheorem}{Theorem}
\newtheorem{innercorollary}{Corollary}
\newtheorem{innerlemma}{Lemma}
\newtheorem{innerproposition}{Proposition}

\theoremstyle{definition}
\newtheorem{innerdefinition}{Definition}
\theoremstyle{plain}

\newenvironment{theorem*}[1]{%
  \renewcommand\theinnertheorem{#1}%
  \innertheorem
}{\endinnertheorem}

\newenvironment{lemma*}[1]{%
  \renewcommand\theinnerlemma{#1}%
  \innerlemma
}{\endinnerlemma}

\newenvironment{corollary*}[1]{%
  \renewcommand\theinnercorollary{#1}%
  \innercorollary
}{\endinnercorollary}

\newenvironment{definition*}[1]{%
  \renewcommand\theinnerdefinition{#1}%
  \innerdefinition
}{\endinnerdefinition}

\newenvironment{proposition*}[1]{%
  \renewcommand\theinnerproposition{#1}%
  \innerproposition
}{\endinnerproposition}

%%% MATH OPERATORS
\DeclareMathOperator{\spn}{span} % linear span
\DeclareMathOperator{\ann}{ann} % annihilator
\DeclareMathOperator{\Var}{Var} % variance
\DeclareMathOperator{\trace}{tr} % variance

\DeclareMathOperator{\NulSp}{Nul} % variance
\DeclareMathOperator{\ColSp}{Col} % variance
\newcommand{\dotp}{\boldsymbol{\cdot}} % dot product \usepackage{algpseudocode,algorithm}
\author{Abraham White}
\date{\today}
\title{Serial SAT Solver}
\hypersetup{
 pdfauthor={Abraham White},
 pdftitle={Serial SAT Solver},
 pdfkeywords={},
 pdfsubject={},
 pdfcreator={Emacs 26.3 (Org mode 9.2.6)}, 
 pdflang={English}}
\begin{document}

\maketitle
\section*{Description}
\label{sec:org0fa1768}
The Boolean Satisfiability Problem (SAT) is an NP-complete decision problem,
where the goal is to find whether there is an assignment of values to boolean variables
in a propositional formula such that the formula evaluates to true. For example,
determining whether \[ a \land \neg a \] can be true for some value of \(a\) is
an SAT problem.

Most SAT solvers take their input as a propositional logic formula in conjunctive normal form (CNF),
involving variables and the operators \emph{negation} (\(\neg\)), \emph{disjunction} (\(\lor\)), and \emph{conjunction} (\(\land\)). A
propositional logic formula in CNF is the conjunction of a set of clauses. A \emph{clause} is a disjunction
of literals, and a \emph{literal} is a boolean variable \(A\) which can be either positive (\(A\)), or negative (\(\neg A\)).

An \emph{interpretation} is a mapping from a CNF formula to the set of truth values \(\{\top,\bot\}\) through assignment
of truth values to the literals in the formul. SAT solvers use \emph{partial interpretations}, where only some of the
literals in the formula are assigned truth values. These variables are replaced with their truth values in
the formula and the formula is then simplified using the rules of propositional logic.

The Boolean Satisfiability Problem is the question of whether there exists an interpretation for a formula such
that the formula evaluates to \(\top\) under this interpretation.

There are two types of SAT solvers: complete, and stochastic. Complete solvers attempt to
either find a solution, or show that no solutions exist. Stochastic solvers cannot prove
that a formula is unsolvable, but can find solutions for specific kinds of problems very
quickly. We are attempting to build a complete SAT solver.

The majority of modern complete SAT solvers are based on a branch and backtracking algorithm called
Davis-Putnam-Logemann-Loveland (DPLL), a refinement of the earlier Davis-Putnam algorithm and 
introduced in 1962 by Martin Davis, George Logemann, and Donald W. Loveland. Many of these solvers
add additional heuristics on top of the DPLL algorithm, which can increase efficiency, but adds significant
complexity to the implementation. Because of this, we will only focus on the DPLL algorithm and refrain
from any heuristical approaches. We use Formalization and Implementation of Modern SAT Solvers
\cite{formalizationImpl} as a guide in the implementation of our serial SAT solver.


\section*{Implementation}
\label{sec:orga5320f4}
\begin{algorithm}
\caption{The recursive DPLL algorithm}
\label{alg:rec_dpll}
\begin{algorithmic}[1]
\Function{dpll}{$F$ : Formula}
\If {$F$ is empty}
  \State \Return SAT
\ElsIf {$F$ contains an empty clause}
  \State \Return UNSAT
\ElsIf {$F$ contains a pure literal $l$} \label{alg:rec_dpll_litelim}
  \State \Return \Call{dpll}{$F[l \to \top]$}
\ElsIf {$F$ contains a unit clause $[l]$} \label{alg:rec_dpll_unitprop}
  \State \Return \Call{dpll}{$F[l \to \top]$}
\Else
  \State let $l$ be a literal in $F$
  \If {\Call{dpll}{$F[l \to \top]$} = SAT}
    \State \Return SAT
  \Else
    \State \Return \Call{dpll}{$F[l \to \bot]$}
  \EndIf
\EndIf
\EndFunction
\end{algorithmic}
\end{algorithm}

The basic DPLL algorithm can be defined recursively as in Algorithm~\ref{alg:rec_dpll}.
In the algorithm, \(F[l \to \top]\) denotes the formula obtained by replacing the literal \(l\) with \(\top\) and
\(\neg l\) with \(\bot\) in \(F\). A literal is pure if it occurs in \(F\) but its opposite does not. A clause is unit
if it contains only one literal. Unfortunately, this recursive algorithm becomes unusable with larger formulae.

The DPLL algorithm consists of two key steps:
\begin{enumerate}
\item \textbf{Literal Elimination}: If some literal is only seen in pure form, we can immediately determine the truth value
for that literal. For instance, if the literal is in the form \(A\), we know that \(A\) must be \(\top\), and if
the literal is in the form \(\neg A\), \(A\) must be \(\bot\). This step occurs on line \ref{alg:rec_dpll_litelim} of the
recursive algorithm .
\item \textbf{Unit Propogation}: If there is a unit clause then we can immediately assign a truth value in the same way we
do for literal elimination. This is done on line \ref{alg:rec_dpll_unitprop} of the recursive algorithm.
\end{enumerate}

\bibliographystyle{unsrt}
\bibliography{refs}
\end{document}