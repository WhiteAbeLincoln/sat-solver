\textheight=9.25in
\textwidth=7in
\topmargin=-0.75in
\oddsidemargin=-0.25in
\evensidemargin=-0.25in
\usepackage{mathtools}
\usepackage{pgfplots}
\usepackage[binary-units]{siunitx}
\usepackage{xstring}
\usepackage{mdframed}
\usepackage{amsthm}
\usepackage{nicematrix}
\NiceMatrixOptions{transparent,nullify-dots}

\allowdisplaybreaks

\usemintedstyle{xcode}

%%% NAMED SET COMMANDS
\newcommand{\RR}{\mathbb{R}}
\newcommand{\NN}{\mathbb{N}}
\newcommand{\ZZ}{\mathbb{Z}}
\newcommand{\QQ}{\mathbb{Q}}
\newcommand{\CC}{\mathbb{C}}
\newcommand{\PP}{\mathbb{P}}

%%% VECTOR COMMAND
\newcommand{\vect}[2][bold]{
  \IfEqCase{#1}{%
    {bold}{\mathbf{#2}}%
    {arrow}{\overrightarrow{#2}}%
  }%
}

\newcommand{\falling}[2]{#1^{\underline{#2}}}

%%% SHORT MATRIX COMMAND
% Prototype: MACRO mat OPT[#1={b}] OPT[#2={r}] #3
\makeatletter
\newcommand{\mat}[1][b]{%
  \@ifnextchar[{\mat@i[{#1}]}{\mat@i[{#1}][{r}]}%
}

\def\mat@i[#1][#2]#3{%
  % Put your code here.
  % You can refer to the arguments as #1 through #3.
  \IfEqCase{#1}{%
    {p}{\begin{pmatrix*}[#2]#3\end{pmatrix*}}%
    {b}{\begin{bmatrix*}[#2]#3\end{bmatrix*}}%
    {B}{\begin{Bmatrix*}[#2]#3\end{Bmatrix*}}%
    {v}{\begin{vmatrix*}[#2]#3\end{vmatrix*}}%
    {V}{\begin{Vmatrix*}[#2]#3\end{Vmatrix*}}%
    {s}{\begin{smallmatrix*}[#2]#3\end{smallmatrix*}}%
    {sp}{\begin{psmallmatrix*}[#2]#3\end{psmallmatrix*}}%
    {sb}{\begin{bsmallmatrix*}[#2]#3\end{bsmallmatrix*}}%
    {sB}{\begin{Bsmallmatrix*}[#2]#3\end{Bsmallmatrix*}}%
    {sv}{\begin{vsmallmatrix*}[#2]#3\end{vsmallmatrix*}}%
    {sV}{\begin{Vsmallmatrix*}[#2]#3\end{Vsmallmatrix*}}%
  }%[\begin{#1}#3\end{#1}]
}
\makeatother

%%% ANSWER ENVIRONMENT
\mdfdefinestyle{answerstyle}{innerrightmargin=10pt,innerleftmargin=10pt,frametitle={Answer:},linewidth=2pt}
\newenvironment{answer}{%
  \begin{mdframed}[style=answerstyle]
  \begingroup
  \setlength{\parindent}{0pt}
}{
  \endgroup
  \end{mdframed}
}
% answer environment with indentation 
\newmdenv[style=answerstyle]{answeri}

%%% EVALUATION BARS
\newcommand{\evald}{\bigg\rvert}
\newcommand{\evaldb}{\Bigg\rvert}
\newcommand{\evalds}{\Big\rvert}

%%% THEOREM ENVIRONMENTS
\newtheorem{theorem}{Theorem}[section]
\newtheorem{corollary}{Corollary}[theorem]
\newtheorem{lemma}[theorem]{Lemma}
\newtheorem{proposition}{Proposition}[section]

\theoremstyle{definition}
\newtheorem{definition}{Definition}[section]
\theoremstyle{plain}

%%% MANUALLY NUMBERED THEOREM ENVIRONMENTS
% don't care about numbering since it must be set manually
\newtheorem{innertheorem}{Theorem}
\newtheorem{innercorollary}{Corollary}
\newtheorem{innerlemma}{Lemma}
\newtheorem{innerproposition}{Proposition}

\theoremstyle{definition}
\newtheorem{innerdefinition}{Definition}
\theoremstyle{plain}

\newenvironment{theorem*}[1]{%
  \renewcommand\theinnertheorem{#1}%
  \innertheorem
}{\endinnertheorem}

\newenvironment{lemma*}[1]{%
  \renewcommand\theinnerlemma{#1}%
  \innerlemma
}{\endinnerlemma}

\newenvironment{corollary*}[1]{%
  \renewcommand\theinnercorollary{#1}%
  \innercorollary
}{\endinnercorollary}

\newenvironment{definition*}[1]{%
  \renewcommand\theinnerdefinition{#1}%
  \innerdefinition
}{\endinnerdefinition}

\newenvironment{proposition*}[1]{%
  \renewcommand\theinnerproposition{#1}%
  \innerproposition
}{\endinnerproposition}

%%% MATH OPERATORS
\DeclareMathOperator{\spn}{span} % linear span
\DeclareMathOperator{\ann}{ann} % annihilator
\DeclareMathOperator{\Var}{Var} % variance
\DeclareMathOperator{\trace}{tr} % variance

\DeclareMathOperator{\NulSp}{Nul} % variance
\DeclareMathOperator{\ColSp}{Col} % variance
\newcommand{\dotp}{\boldsymbol{\cdot}} % dot product